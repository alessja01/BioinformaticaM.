\documentclass{report}
\usepackage[italian]{babel}
\usepackage{amsmath}

\begin{document}
\chapter{Allineamento multiplo di sequenze}
\section{Contesto Generale e nascita sequence analysis}
La rivoluzione nell'analisi genetica, iniziata negli anni 70 con lo
sviluppo della tecnologia del DNA ricombinate e delle tecniche di 
sequenziamento, ha portato alla disponibilità di un'enorme quantità di dati
biologici. Oggi esistono vasti database contenenti sequenze di nucleotidi e
amminoacidi provenienti da una varietà di organismi. Nonostante questa abbondanza d'informazioni
di sequenza, per una larga frazione dei geni e delle proteine annotate
non sono note nè la struttura tridimensionale nè la funzione biologica.
Questo ha portato alla nascita di un nuovo campo di ricerca, noto come
sequence analysis, che si occupa di sviluppare metodi computazionali per
l'analisi delle sequenze biologiche al fine di inferire informazioni strutturali
e funzionali. La sequence analysis includono diverse fasi a seconda dell'origine
della sequenza e degli obiettivi dell'analisi. Tra queste fasi, una delle più importanti
e ricorrenti è \textbf{l'allineamento di sequenze}, utilizzato per individuare omologie e per confrontare
sequenze nuove con quelle già presenti nei database.

\section{Ruolo dell'Allineamento di sequenze}
L'allineamento di sequenze è un passaggio chiave sia per: il confronto 
diretto tra due o più sequenze; sia per il database searching, ovvero 
la ricerca di sequenze simili in grandi archivi biologici.
L'identificazione di sequenze simili ha numerose applicazioni:
\begin{itemize}
    \item Permette di riconoscere geni codificanti all'interno di sequenze genomiche grezze 
    \item Consente di assegnare una funzione putativa a un gene tramite il confronto con geni già caratterizzati
    \item Fornisce indicazioni sulla struttura tridimensionale delle proteine 
\end{itemize}

\section{Concetto di Allineamento e introduzione dei Gap }
L'allineamento di sequenze consiste nel disporre due o più sequenze in modo tale che le residui 
derivati da uno stesso residuo ancestrale occupino la stessa posizione. Questo obiettivo viene 
generalmente preseguito massimizzando la similarità complessiva dell'allineamento. Quando le
sequenze differiscono in lunghezza, un allineamento diretto produce:
\begin{itemize}
    \item Posizioni in cui i residui non corrispondono (mismatches)
    \item Posizioni in cui una sequenza ha residui aggiuntivi rispetto all'altra (inserzioni o delezioni, note come indels)
\end{itemize}
L'introduzione di \textbf{gap} che rappresentano insezioni o delezioni evolutive, consente di migliorare l'allineamento tra le sequenze.
Tuttavia non esistono regole rigide per l'inserimento dei gap, e la loro gestione rappresenta una sfida significativa nell'allineamento di sequenze.
Per questo motivo, gli algoritmi di allineamento devono :
\begin{itemize}
    \item generare possibili allineamenti
    \item assegnare a ciascuno un punteggio quantitativo
    \item scartare quelli non significarivi in base a criteri statistici 
\end{itemize}

\section{Similarità a Confronto con omologia}
Un punto concettuale fondamentale è la distinzione tra similarità e omologia.
La Similarità è un termine desccrittivo che indica un certo grado di corrispondenza tra sequenze; 
L'omologia implica invece una relziione evolutiva, ovvero la derivazione da un antenato comune
Due sequenze omologhe possono aver divergenze significative a livello di sequenza, pur mantenendo :
\begin{itemize}
    \item una struttura tridimensionale simile
    \item una funzione biologica analoga
\end{itemize}
Tuttavia: un elevata similarità tra sequenze non garantisce necessariamente un'origine evolutiva comune; ed una bassa
similarità non esclude l'omologia. Esistono anche casi di evoluzione convergente, in cui sequenze non omologhe
mostrano somiglianze locali dovute a vincoli funzionali simili. In tali casi, la similarità 
non riflette una comune origine evolutiva. Un allineamento deve quindi essere interpretato come una 
ipotesi sulle corrispondenze evolutive tra residui, non come una dimostrazione definitiva di omologia.
Esistono anche casi di \textbf{evoluzione convergente}, in cui sequenze non omologhe
mostrano somiglianze locali dovute a vincoli funzionali simili. In tali casi, la similarità 
non riflette una comune origine evolutiva. Un allineamento deve quindi essere interpretato come una 
ipotesi sulle corrispondenze evolutive tra residui, non come una dimostrazione definitiva di omologia.

\section{Modelli evolutivi, scoring e algoritmi}
I metodi computazioni di confronto delle sequenze devono tenere conto di :
\begin{itemize}
    \item diversi tipi di mutazione
    \item proprietà fisico-chimiche degli amminoacidi
    \item pressioni selettive che favoliscono o eliminano determinate variazioni
\end{itemize}
Questi fattori vengono incorporati in :
\begin{itemize}
    \item \textbf{schemi di scoring} che assegnano punteggi a matches, mismatches e gap
    \item \textbf{algoritmi di allineamento} che cercano di ottimizzare il punteggio complessivo 

\end{itemize}
Infine, è essenziale distiguere tra 
\begin{itemize}
    \item allineamenti apparentemente buoni ma dovuti al caso
    \item allinemaneti che riflettono una relazione evolutiva reale, utilizzando criteri statici adeguati.
\end{itemize}

\section{Limiti della percent identity come misura di similarità}
La \textbf{percentuale d'identità} rappresenta la misura più semplce e 
immediata della quantità di un allineamento, in quanto si ottiene 
semplicemente calcolando la frazione di posizioni allineate in cui 
i residui sono identici. Sebbene questa misura sia utile come 
\textbf{test preliminare rapido}, essa risulta concettualmente grossolana
e insufficiente a descrivere in modo accurato il grado reale di somiglianza
biologica tra due sequenze, in particolare nel caso delle \textbf{sequenze proteiche}
Il principale limite della percent identity risiede nel fatto che essa assegna un valore
binario alle posizioni allineate:
\begin{itemize}
    \item \textbf{1} per un match identico 
    \item \textbf{0} per un mismatches
\end{itemize}
Questo approccio ignora completamente la natura chimica e fisica degli
amminoacidi coinvolti, trattando allo stesso modo sostituizioni biologicamente
plausibili e sostituzioni estremamente improbabili dal punto di vista evolutivo.

\section{Similarità funzionale tra amminoacidi non identici}





\end{document}

